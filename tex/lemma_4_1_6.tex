\documentclass{article}
\usepackage{xeCJK}
\usepackage{theorem}
\usepackage{amsmath}
\newtheorem{theorem}{Theorem}
\newtheorem{lemma}{Lemma}
\newtheorem{proof}{Proof}[section]
\title{Lemma 4.1.6}
\author{Clexma}
\begin{document}
\maketitle
\begin{lemma}[4.1.6] A flow $f$ is a \textit{s-t} path flow iff $f$ satisfies the following conditions:
\begin{itemize}
\item If s = t then
\begin{enumerate}
\item $\Sigma_{w\in out(v)} f(v,w) = \Sigma_{w\in in(v)} f(w,v)$ for all v
\item $F(f)$ is a s-t support


\end{enumerate}


\item If s $\ne$ t then
\begin{enumerate}
\item $\Sigma_{w\in out(v)} f(v,w) = \Sigma_{w\in out(v)} f(w,v)$ for all $v\in V-\{s,t\}$

$\Sigma_{w\in out(s)} f(s,w) = \Sigma_{w\in out(s)} f(w,s) + 1$

$\Sigma_{w\in out(t)} = \Sigma_{w\in out(t)} f(w,t) - 1$


\item F(f) is connected.
\end{enumerate}

\end{itemize}



\end{lemma}


\begin{proof} Prove by induction on $n  = \Sigma_{e\in E} f(e)$\\
\begin{itemize}
\item $\Rightarrow$:
 By definition direction of path flow the proof is obvious.

\item $\Leftarrow$:
\begin{enumerate}
\item $s = t, F(f) is connected$, we have the conclusion that $f$ is a s-t path flow.

\begin{itemize}
\item case 1 : there is a self loop from $s$ to $s$.

\item case 2 : there is no self loop from $s$ to $s$. In this case we delete an edge from $v$ to $s$ such that $v \neq s$.

\begin{itemize}
\item subcase 2.1 the resulting graph is still connectedl let the new flow be $f'$. By the induction hypothesis $f'$ is a path flow from $s$ to $v$.

\item subcase 2.2 the resulting graph is not connected, in this case , there are exactly two connected components such in the resulting graph, we call them $C_1$ and $C_2$ and assume they contain $s$ and $v$ respectively.
\begin{itemize}
\item sub-subcase 2.2.1:
$C_2$ contains no edges.

\item sub-subcase 2.2.2:
$C_1$ and $C_2$ both contain at least one edge.

\end{itemize}
\end{itemize}
\end{itemize}

\end{enumerate}
\end{itemize}


\end{proof}
\end{document}